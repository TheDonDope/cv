% Copyright 2013 Christophe-Marie Duquesne <chmd@chmd.fr>
% Copyright 2014 Mark Szepieniec <http://github.com/mszep>
%
% ConText style for making a resume with pandoc. Inspired by moderncv.
%
% This CSS document is delivered to you under the CC BY-SA 3.0 License.
% https://creativecommons.org/licenses/by-sa/3.0/deed.en_US

\startmode[*mkii]
  \enableregime[utf-8]
  \setupcolors[state=start]
\stopmode
%\mainlanguage[de]

\setupcolor[hex]
\definecolor[titlegrey][h=757575]
\definecolor[sectioncolor][h=397249]
\definecolor[rulecolor][h=9cb770]

% Enable hyperlinks
\setupinteraction[state=start, color=sectioncolor]

\setuppapersize [A4][A4]
\setuplayout    [width=middle, height=middle,
                 backspace=20mm, cutspace=0mm,
                 topspace=10mm, bottomspace=20mm,
                 header=0mm, footer=0mm]

%\setuppagenumbering[location={footer,center}]

\setupbodyfont[11pt, helvetica]

\setupwhitespace[medium]

\setupblackrules[width=31mm, color=rulecolor]

\setuphead[chapter]      [style=\tfd]
\setuphead[section]      [style=\tfd\bf, color=titlegrey, align=middle]
\setuphead[subsection]   [style=\tfb\bf, color=sectioncolor, align=right,
                          before={\leavevmode\blackrule\hspace}]
\setuphead[subsubsection][style=\bf]

\setuphead[chapter, section, subsection, subsubsection][number=no]

%\setupdescriptions[width=10mm]

\definedescription
  [description]
  [headstyle=bold, style=normal,
   location=hanging, width=18mm, distance=14mm, margin=0cm]

\setupitemize[autointro, packed]    % prevent orphan list intro
\setupitemize[indentnext=no]

\setupfloat[figure][default={here,nonumber}]
\setupfloat[table][default={here,nonumber}]

\setuptables[textwidth=max, HL=none]

\setupthinrules[width=15em] % width of horizontal rules

\setupdelimitedtext
  [blockquote]
  [before={\setupalign[middle]},
   indentnext=no,
  ]


\starttext

\startsectionlevel[title={Christian Dobert
},reference={christian-dobert}]

\startsectionlevel[title={Persönliche
Daten},reference={persönliche-daten}]

\startdescription{{\bf Geboren}}
  10.03.1987
\stopdescription

\startdescription{{\bf Geburtsort}}
  Wippra
\stopdescription

\startdescription{{\bf Familienstand}}
  geschieden
\stopdescription

\startdescription{{\bf Nationalität}}
  deutsch
\stopdescription

\stopsectionlevel

\startsectionlevel[title={Schulbildung},reference={schulbildung}]

\startdescription{2000 - 2006}
  {\bf Allgemeine Hochschulreife}; Geschwister Scholl Gymnasium
  (Sangerhausen)
\stopdescription

\stopsectionlevel

\startsectionlevel[title={Ausbildung},reference={ausbildung}]

\startdescription{September 2006 - Juni 2009}
  {\bf Ausbildung zum Fachinformatiker Anwendungsentwicklung};
  IT-Systemhaus der Bundesagentur für Arbeit (Nürnberg)
\stopdescription

\stopsectionlevel

\startsectionlevel[title={Berufserfahrung},reference={berufserfahrung}]

{\bf Lead Entwickler Online Services (Fullstack), ALTEN Group}

{\em Seit Mai 2023}

Im Mai 2023 fand ein Projektwechsel in eine neue Abteilung im
IT-Systemhaus der Bundesagentur für Arbeit statt, welches für die
Fachverfahren Ausländerkerndatensystem (AKDS) und X-AUSLÄNDER
verantwortlich ist. Diese Fachverfahren bieten Schnittstellen zum
Ausländerzentralregister (AZR) um Daten zu Integrationsmassnahmen wie
z.B. Deutschkursen auszutauschen.

Die technische Aufgabenstellung war auch hierbei im Zuge der
Gesamtumstellung der IT-Systeme der BA auf Containerplattform zu
unterstützen. Technologisch wurden hierbei Apache Kafka, Java, Docker
und Kubernetes eingesetzt. Auf Frontendseite kam Angular zum Einsatz.
Die CI/CD Pipeline wurde mit GitHub Enterprise und GitHub Actions
realisiert.

Neben den technischen Tätigkeiten war ein weiteres Arbeitsfeld für mich
die Konzeptarbeit im Security Bereich, welche als nichtfunktionale
Anforderungen vor Produktivsetzung erfüllt werden mussten.

{\bf Senior Consultant Java, ALTEN Group}

{\em Seit Juni 2019}

Nach Abschluss des Projekts APOLLO erfolgte der nahtlose Übergang zum
Nachfolgeprojekt OPO (Online Programm Organisation). Dieses fasst alle
aus dem Vorgängerprojekt entstandenen Onlineprodukte des Bereiches
eGovernment im Kundenkanal der BA zusammen. Ich arbeite hierbei im Team
LEONARDO, welches die Verantwortung für die folgenden Produkte hat:

\startitemize[packed]
\item
  Berufsausbildungsbeihilfe
\item
  Kundenbescheide
\item
  Veränderungsmitteilungen
\item
  Kurzarbeitergeld
\stopitemize

Das Hauptaufgabenfeld des Teams liegt dabei in der technischen
Überführung der Produkte von der alten ADF und Oracle Weblogic
Technologie hin zur Containerplattform auf Basis von DC/OS und Docker.
Hierbei werden die Produkte in Angular als Frontend- und Spring Boot als
Backendtechnologie implementiert. Im Team übernehme ich dabei
nachfolgende Tätigkeiten:

\startitemize[packed]
\item
  Frontendentwicklung mit Angular
\item
  Backendentwicklung mit Spring Boot
\item
  IT-Sicherheitsverantwortlicher des Teams, hierbei Auswertung von
  NexusIQ Reports, Planung und Behebung von auftretenden
  Sicherheitslücken
\item
  Betreuung und Deployment des containerisierten Kundenbescheide
  Produkts (via Jenkins, Docker, DC/OS)
\item
  Überwachung des Produktionsbetriebs des containerisierten
  Kundenbescheide Produkts (via Kibana)
\stopitemize

{\em Seit Oktober 2018}

Seit Oktober 2018 unterstütze ich im IT-Systemhaus der Bundesagentur für
Arbeit bei der Umsetzung der Online Kunden Kanäle.

Neben der Frontendentwicklung verschiedener Antragsstrecken
(Berufsausbildungsbeihilfe, Kurzarbeitergeld, Insolvenzgeld),
Veränderungsmitteilungen und Kundenbescheide liegt der aktuelle Fokus
auf der Containerisierung verschiedener Module mit Docker und DC/OS.
Zusätzlich nehme ich die Aufgabe des Team IT-Sicherheitsverantwortlichen
wahr.

In diesem agilen Projektumfeld verantworten wir ebenso den
teamübergreifenden Lieferprozess für das Portal Lieferpaket.

{\bf Lead Consultant Java, ALTEN Group}

{\em November 2016 - September 2018}

Neben meinem Einsatz in Kundenprojekten, speziell im öffentlichen
Dienst, bin ich seit September 2016 Fachausbilder für die Ausbildung zum
Fachinformatiker Anwendungsentwicklung.

Beginnend mit dem Ausbildungsjahr 2006 befinden sich derzeit drei
Auszubildende im ersten Lehrjahr der Ausbildung. Besonderen Wert legen
wir auf die Vermittlung aktuellster Technologiethemen um einen
bestmöglichen Start in das Berufsleben zu bieten.

Zusätzlich dazu arbeite ich mich seit über einem Jahr in das
Webframework \goto{Angular}[url(https://angular.io/)] ein, welches ich
schon in einigen Projekten zum Einsatz bringen konnte. In Verbindung mit
dem Einsatz von Docker wage ich den Blick über den typischen
Softwareentwickler Tellerrand und erprobe für mich auch das DevOps
Gebiet.

Meine eigenen Lernerfolge teile ich gern und unterstütze aktiv in Open
Source Projekten. Eine Auswahl meiner Aktivitäten finden sie im
Abschnitt \goto{Open Source}[open-source].

{\bf Java Enterprise Entwickler, ALTEN Group}

{\em Januar 2014 - Oktober 2016}

Mit meinem Wechsel in die freie Wirtschaft ging nicht nur ein einfacher
Arbeitgeber-, sondern auch ein Kulturwechsel einher. Als Consultant
entwickelte ich vor Ort im Kundenprojekt Software für einen der
wichtigsten Sozialversicherungsträger. Dem Begriff
\quotation{Consultant} sollte auch Rechnung getragen werden, und so
gehörten zu meinen Tätigkeiten nicht nur reine Softwareentwicklung.
Architekturentwürfe, Design-Spezifikationen, Einarbeitung neuer
Mitarbeiter, Wissensvermittlung und Teamleitung gehörten seither zu
meinem Tätigkeitsfeld.

{\bf IT-Techniker Design & Implementierung, IT-Systemhaus der
Bundesagentur für Arbeit, Nürnberg}

{\em August 2011 - Dezember 2013}

Auch der öffentliche Dienst kann sich nicht völlig von der Außenwelt
abschotten, und so ereilte sich auch bei der Bundesagentur für Arbeit
der Zeitpunkt bei welchem das Client-Betriebssystem Windows XP erneuert
werden musste. Die Migration zu Windows 7 lief über einen mehrphasigen
und mehrjährigen Prozess, bei welchem eine Vielzahl von
Herausforderungen bewältigt werden mussten. Ein überarbeitetes Benutzer
und Berechtigungskonzept, neue Ordnerstrukturen, strengere
Sicherheitsrichtlinien; all diese und noch viel mehr Themen hatten
direkten oder indirekten Einfluss auf die Entwicklung des Java Swing
Clients der Allgemeinen Terminverwaltung. Nach langer Entwicklungszeit,
vielen gefällten Architekturentscheidungen und Lösungsentwürfen, konnte
die Migration erfolgreich und termingerecht fertiggestellt werden.

{\bf IT-Fachassistent Design & Implementierung, IT-Systemhaus der
Bundesagentur für Arbeit, Nürnberg}

{\em Juli 2009 - Juli 2011}

Nach dem erfolgreichen Abschluss meiner Ausbildung zum Fachinformatiker
Anwendungsentwicklung begann ich meinen Berufsweg im Verfahren
Allgemeine Terminverwaltung (ATV). Mit einer Nutzerzahl von 30.000
täglichen parallelen Nutzern, hunderttausenden vereinbarten Terminen und
einem Einladungsdruckvolumen von über einer Millionen Briefen pro Monat
war dies eines der elementaren Basisverfahren der Bundesagentur für
Arbeit.

In meinem Tätigkeitsfeld war ich Alleinverantwortlich für die
Weiterentwicklung und Instandhaltung des Java Swing Clients, welcher
eine Vielzahl von JGoodies Bibliotheken nutzt. Neben dem Design von
Oberflächen und deren Implementierung, unterstützte ich unsere Tester
bei der Testautomatisierung per Silk Test, stellte eine barrierefreie
Umsetzung der Software sicher und bearbeitete Incidents von Anwendern im
Second Level Support.

\stopsectionlevel

\startsectionlevel[title={Qualifikationen},reference={qualifikationen}]

{\bf Oracle Certified Associate, Java SE 8 Programmer}

{\em Juli 2016}

\startitemize[packed]
\item
  \goto{Zertifikat
  Link}[url(https://www.youracclaim.com/badges/9fd343c0-b445-4663-815e-5356c5c82f27/linked_in_profile)]
\stopitemize

{\bf Professional Scrum Master I (scrum.org)}

{\em Oktober 2014}

\startitemize
\item
  License 88699
\item
  \goto{Zertifikat Link}[url(https://www.scrum.org/user/159323)]
\stopitemize

\stopsectionlevel

\startsectionlevel[title={Open Source},reference={open-source}]

\startitemize
\item
  \goto{Arctic Code Vault
  Contributor}[url(https://archiveprogram.github.com/)]:

  \startitemize[packed]
  \item
    Mitarbeit an mehreren Angular Seed Projekten
    (https://github.com/TheDonDope?achievement=arctic-code-vault-contributor&tab=achievements)
  \stopitemize
\item
  Eigene Projekte:

  \startitemize
  \item
    \useURL[url1][https://github.com/TheDonDope/crudular]\from[url1]

    \startitemize[packed]
    \item
      An Angular CRUD example application using JWT and consuming a JSON
      API
    \stopitemize
  \item
    \useURL[url2][https://github.com/TheDonDope/gordle]\from[url2]

    \startitemize[packed]
    \item
      A golang TUI implementation of the popular word quiz wordle!
    \stopitemize
  \item
    \useURL[url3][https://github.com/TheDonDope/cv]\from[url3]

    \startitemize[packed]
    \item
      My personal curriculum vitae
    \stopitemize
  \stopitemize
\stopitemize

\stopsectionlevel

\startsectionlevel[title={Kenntnisse und
Hobbys},reference={kenntnisse-und-hobbys}]

\startitemize
\item
  Sprachkenntnisse:

  \startitemize[packed]
  \item
    Deutsch (Muttersprache)
  \item
    English (Verhandlungssicher)
  \item
    Russisch (Grundkenntnisse)
  \stopitemize
\item
  Führerscheine:

  \startitemize[packed]
  \item
    B
  \stopitemize
\item
  Softwarekenntnisse:

  \startitemize[packed]
  \item
    Microsoft Office Suite
  \item
    verschiedenste IDEs (\goto{Eclipse}[url(http://www.eclipse.org/)],
    \goto{IntelliJ}[url(https://www.jetbrains.com/idea/)],
    \goto{Webstorm}[url(https://www.jetbrains.com/webstorm/)])
  \item
    verschiedenste Editoren (\goto{Sublime
    Text}[url(https://www.sublimetext.com/)],
    \goto{Atom}[url(https://atom.io/)], \goto{Visual Studio
    Code}[url(https://code.visualstudio.com/)])
  \item
    \goto{Innovator}[url(http://www.mid.de/en/business-activities/tools/innovator)]
  \item
    \goto{Star UML}[url(http://staruml.io/)]
  \stopitemize
\item
  Entwicklungswerkzeuge:

  \startitemize[packed]
  \item
    \goto{Docker}[url(https://www.docker.com/)]
  \item
    \goto{NodeJS}[url(https://nodejs.org/en/)]
  \item
    \goto{nginx}[url(https://nginx.org/en/)]
  \item
    \goto{Express}[url(https://expressjs.com/)]
  \item
    \goto{PostgreSQL}[url(https://www.postgresql.org/)]
  \item
    \goto{MongoDB}[url(https://www.mongodb.com/)]
  \stopitemize
\item
  Programmiersprachen:

  \startitemize
  \item
    Java

    90\letterpercent{}
  \item
    Go

    80\letterpercent{}
  \item
    Typescript

    80\letterpercent{}
  \item
    HTML5

    80\letterpercent{}
  \item
    CSS3

    80\letterpercent{}
  \item
    Ruby

    70\letterpercent{}
  \item
    Python

    60\letterpercent{}
  \stopitemize
\item
  Hobbys:

  \startitemize[packed]
  \item
    \goto{Musikproduktion}[url(https://soundcloud.com/thedondope)]
  \item
    Open Source Entwicklung:
    \startitemize[packed]
    \item
      \goto{GitHub}[url(https://github.com/TheDonDope)]
    \item
      \goto{GitLab}[url(https://gitlab.com/TheDonDope)]
    \stopitemize
  \item
    Bouldern
  \stopitemize
\stopitemize

\thinrule

\startblockquote
\goto{chr.dobert@gmail.com}[url(mailto:chr.dobert@gmail.com)] • +49 176
70784300 • 37 Jahre\crlf
Bernstädter Strasse 9, 90473 Nürnberg
\stopblockquote

\stopsectionlevel

\stopsectionlevel

\stoptext
